\documentclass[aspectratio = 169]{beamer}
%\documentclass[handout, notes = show]{beamer} %Para mostrar notas. 

%%%%%%%%%%%%%%%%%%%%%%%%%%%%%%%%%%%%%%%%%%%%%%%
%                                             %
%             Load Packages                   %
%                                             %
%%%%%%%%%%%%%%%%%%%%%%%%%%%%%%%%%%%%%%%%%%%%%%%

\usetheme{default}
\usepackage{amsmath} % NdR: for \begin{align} to work.
\usepackage{bbm} % NdR: needed indicator function symbol
\usepackage[utf8]{inputenc} % NdR: tildes en español. 
\usepackage{adjustbox} % NdR: to insert big tables into presentations. From .tex of Margarita Gáfaro.
\setbeamertemplate{navigation symbols}{} 
\setbeamertemplate{itemize items}[circle]
\setbeamertemplate{itemize subitem}{$-$}
\usefonttheme[onlymath]{serif}

\usepackage{graphicx} % NdR: varios de gráficos. 
\usepackage{multirow} % NdR: para multirows in tables.
\usepackage{comment}  % NdR: para comentarios de secciones enteras de texto.
\usepackage{booktabs} % NdR: para el comando \cmidrule que facilita hacer lineas horizontales en las tablas sin que se toquen.
\usepackage{threeparttable} %NdR: for tables (and table notes).
\usepackage{xcolor} %NdR: para definir colores y colorear ecuaciones. 
\usepackage{caption}
\usepackage{subcaption}
	
	
%%%%%%%%%%%%%%%%%%%%%%%%%%%%%%%%%%%%%%%%%%%%%%%
%                                             %
%              Slides Settings                %
%                                             %
%%%%%%%%%%%%%%%%%%%%%%%%%%%%%%%%%%%%%%%%%%%%%%%	
	
%--- NdR: covered parts of slide displayed transparently.
% \setbeamercovered{transparent} 
%\setbeamercovered{invisible}

% %--- NdR: definine blue	
% \definecolor{azulndr}{RGB}{52,113,206} 
	
% %--- NdR: set color of slides
% \setbeamercolor*{title}{bg=white,fg=azulndr}
% \setbeamercolor*{item}{fg= azulndr}
% \setbeamercolor{frametitle}{fg=azulndr}
	
% %--- NdR: Set color of slide buttons
% \setbeamercolor{button}{fg= white, bg=  azulndr}
% \renewcommand{\beamerreturnbutton}[1]{%
% 	\begingroup% keep color changes local
% 	\setbeamercolor{button}{fg= white , bg= azulndr}%
% 	\beamerbutton{\insertreturnsymbol#1}% original definition
% 	\endgroup
% }
	
%--- NdR : Change the margins of one of the pages:
%  from http://stackoverflow.com/questions/1670463/latex-change-margins-of-only-a-few-pages
%  To use: \begin{changemargin}{-1cm}{-1cm} don't forget to \end{changemargin}
\newenvironment{changemargin}[2]{%
	\begin{list}{}{%
			\setlength{\topsep}{0pt}%
			\setlength{\leftmargin}{#1}%
			\setlength{\rightmargin}{#2}%
			\setlength{\listparindent}{\parindent}%
			\setlength{\itemindent}{\parindent}%
			\setlength{\parsep}{\parskip}%
		}%
	\item[]}{\end{list}}
	
%--- NdR: to change margin of all the frames:
\setbeamersize{text margin left= 0.2in, text margin right= 0.2in}

	
% --- NdR: Put number to slide
	\addtobeamertemplate{navigation symbols}{}{%
		\usebeamerfont{footline}%
		\usebeamercolor[fg]{footline}%
		\hspace{1em}%
		\insertframenumber/\inserttotalframenumber}
	\addtocounter{page}{-1}

% --- Restart counter for appendix slides
\newcommand{\backupbegin}{
   \newcounter{framenumberappendix}
   \setcounter{framenumberappendix}{\value{framenumber}}
}
\newcommand{\backupend}{
   \addtocounter{framenumberappendix}{-\value{framenumber}}
   \addtocounter{framenumber}{\value{framenumberappendix}} 
}

% To shade gray: deemphasize text
\newcommand{\light}[1]{\textcolor{gray!50!white}{#1}}

\newcommand{\sym}[1]{\ifmmode^{#1}\else\(^{#1}\)\fi}

%%%%%%%%%%%%%%%%%%%%%%%%%%%%%%%%%%%%%%%%%%%%%%%
%                                             %
%              Title Page                     %
%                                             %
%%%%%%%%%%%%%%%%%%%%%%%%%%%%%%%%%%%%%%%%%%%%%%%

 \vspace*{-0.5cm}

\title[]{Shocks in Latin American Household Surveys: \\ Incidence and Consequences}

 \vspace*{0.1cm}

%  \author[Luis Martinez]{}
%  \institute[Emory]{\small{\begin{tabular}{c c c c c}
% %  \textbf{Julián Arteaga} & \textbf{Nicolás de Roux} & \textbf{Magarita Gáfaro} & \textbf{Ana M. Ibáñez} & \textbf{Heitor Pellegrina} \\
% %  World Bank & Los Andes & BanRep & IDB & Notre Dame\\ \\
%  \end{tabular}}}
 
\date{
	% \small{\small Universidad del Rosario \\ 
	
	% \vspace{0.5cm}
	April 2025}

%%%%%%%%%%%%%%%%%%%%%%%%%%%%%%%%%%%%%%%%%%%%%%%
%                                             %
%                  Slides                     %
%                                             %
%%%%%%%%%%%%%%%%%%%%%%%%%%%%%%%%%%%%%%%%%%%%%%%

\begin{document}

\begin{frame}
  \titlepage
\end{frame}
\begin{frame}{Research Question}

	\begin{itemize}
  
	  \item \textbf{Big picture:} How do households cope with different types of shocks?
		\begin{itemize}
		  \item What types of government policies can foster mitigation?
		\end{itemize}
		\bigskip
  
	  \item \textbf{Literature:}
		\begin{itemize}
  
		  \item Large literature on impact of temporary vs.\ permanent shocks:
			\begin{itemize}
			  \item \textcolor{gray}{Deaton (1991); Paxson (1992); Townsend (1994) \dots}
			\end{itemize}
			\smallskip
  
		  \item Consumption and income smoothing – Poverty traps:
			\begin{itemize}
			  \item \textcolor{gray}{Rosenzweig \& Wolpin (1993); Morduch (1995); Carter \& Zimmerman (2001)}
			\end{itemize}
			\smallskip
  
		  \item Impact of weather shocks / natural disasters:
			\begin{itemize}
			  \item \textcolor{gray}{Ibáñez et al.\ (2024); Deryugina et al.\ (2018); Kocornik-Mina (2020) \dots}
			\end{itemize}
			\smallskip
  
		  \item Impact of health shocks:
			\begin{itemize}
			  \item \textcolor{gray}{Fadlon \& Nielsen (2021); Mohanan (2013); Garcia-Gomez et al.\ (2013)}
			\end{itemize}
			\smallskip
  
		  \item Policy tools:
			\begin{itemize}
			  \item \textcolor{gray}{de Janvry et al. (2016): \emph{Progresa}}; \textcolor{gray}{Pople et al. (2021): Anticipatory transfers (drought)}; \textcolor{gray}{Carter \& Jansen (2019): Insurance (drought)}
			\end{itemize}
  
		\end{itemize}
  
	\end{itemize}
  
  \end{frame}
  
  
  \begin{frame}{Empirical Approach}
	\begin{itemize}
	  \item \textbf{Descriptive:} Measure prevalence and household response to shocks using household surveys.
	  \bigskip 
	  
	  \item \textbf{Causal impact of weather shocks:} Match remote sensing and meteorological data with household panel data.
	  \bigskip 

	  \item \textbf{Causal impact of idiosincratic shocks:}  administrative data (SISBEN, PILA, RIPS) to identify exogenous variation in shock exposure.
	\end{itemize}
  \end{frame}
  
  \begin{frame}{Data}
	\begin{itemize}
	  \item \textbf{Longitudinal Household Surveys:}
	  \begin{itemize}
		\item ELCA (Colombia)
		\item MxFLS (Mexico)
		\item ENAHO (Peru)
		\item ENAHO (Bolivia)
		\item PNAD-C (Brazil)
		\item EHPM (El Salvador)
		\item EPHPM (Honduras)
	  \end{itemize}
	  \bigskip

	  \item \textbf{Administrative Data:} SISBEN, PILA, RIPS
	  \bigskip 

	  \item \textbf{Remote sensing \& meteorological data} 
	  \begin{itemize}
			\item e.g. \emph{SALUBRAL} dataset
	  \end{itemize}
	\end{itemize}
  \end{frame}
  
  \begin{frame}{Preliminary Hypotheses}
	\begin{itemize}
	  \item Hazard rate of shocks is higher for poorer households.
	  \medskip
	  \item Poorer households have fewer coping mechanisms.
	  \bigskip
	  \begin{itemize}
	  	\item Can we substantiate these claims empirically?
	  	\medskip
	  	\item Can we estimate the impact of government programs as mitigation tools?
	  \end{itemize}
	\end{itemize}
  \end{frame}

\begin{frame}[c]{}         
	\centering
	\large Incidence of self-reported shocks: 
\end{frame}

\begin{frame}{Self-reported shocks: ELCA}

	\begin{figure}
		\includegraphics[scale=0.3]{out/elca_shocks_incidence.png}
   \end{figure}   

\end{frame}

\begin{frame}{Self-reported shocks: rural vs urban}

	\begin{figure}
		\includegraphics[scale=0.3]{out/elca_shocks_incidence_ruralurban.png}
   \end{figure}   

\end{frame}

\begin{frame}{Self-reported shocks: rural vs urban}

	\begin{figure}
		\includegraphics[scale=0.3]{out/elca_anyshock_incidence.png}
   \end{figure}   

\end{frame}

\begin{frame}{Self-reported shocks by consumption quintiles}

	\begin{figure}
		\includegraphics[scale=0.3]{out/elca_anyshock_baselinecons_ruralurban.png}
   \end{figure}   

\end{frame}

\begin{frame}{Self-reported shocks by consumption quintiles - Rural}

	\begin{figure}
		\includegraphics[scale=0.3]{out/elca_shocks_baselinecons_rural.png}
   \end{figure}   

\end{frame}

\begin{frame}{Self-reported shocks by consumption quintiles - Urban}

	\begin{figure}
		\includegraphics[scale=0.3]{out/elca_shocks_baselinecons_urban.png}
   \end{figure}   

\end{frame}

\begin{frame}{Self-reported shocks by consumption quintiles - Average all waves}

	\begin{figure}
		\includegraphics[scale=0.3]{out/elca_anyshock_baselinecons_allysurban.png}
   \end{figure}   

\end{frame}

\begin{frame}[c]{}        
	\centering
	\large Response to shocks:
\end{frame}
  
\begin{frame}{Different responses to by shock type}

	\begin{align*} 
	    \Delta y_{i, t} = \beta Shock_{i,t} + \delta_{t} + \alpha \mathbbm{1}(rural_{i,t-1}=1) + \varepsilon_{i,t}
	\end{align*}

 	\centering
 	\scalebox{0.8}{\input{out/shock_response_elca_13_16.tex}}
 	\medskip

\end{frame}

\begin{frame}{Different responses by household type}

	\vspace{-20pt}
	\begin{align*} 
	    \Delta y^{q}_{i, t} = \beta Shock_{i,t} + \delta_{t} + \alpha \mathbbm{1}(rural_{i,t-1}=1) + \varepsilon_{i,t}
	\end{align*}
	
	\vspace{-10pt}
   \begin{figure}
		\includegraphics[scale=0.3]{out/elca_migrate_natdisast_baselinecons.png}
   \end{figure}   

\end{frame}

\end{document}




% % %
% % %\begin{frame}{Mechanisms: Land Sales}
% % %	Consumption Smoothing
% % %	\begin{itemize}
% % %		\item Lower increases in mortages in poorer
% % %		\item Larger effect on land sales in low density municipalities 
% % %	\end{itemize}
% % %	
% % %\end{frame}
 
% \begin{frame}{Result 2: Shocks reduce the average farm size}\label{sli:effsize}
	
% 	\begin{center}
% 		\scalebox{1}{\input{Tables/t_eff_farmsize}}
% 	\end{center}
	
% \bigskip 
% \vspace{-10pt}
% \flushright
% \hyperlink{sli:apxeffownerslags}{\beamerbutton{Owners lags}}

% \hyperlink{sli:apxeffsizelags}{\beamerbutton{Size lags}} 
% \pause 

% \flushleft
% \begin{itemize}
%     \item Not informative on \emph{which} farms are becoming smaller...
% \end{itemize}

% \end{frame}

% \begin{frame}{Change in number of owners by size bins}

% \flushright
% Drop in average farm size could be because...
% \vspace{-5pt}
% \flushleft
%    \begin{figure}
% 		\includegraphics[scale=0.4]{Figures/initdist_example_0.png}
%    \end{figure}   
% \end{frame}

% \begin{frame}[noframenumbering]{Change in number of owners by size bins}

% \flushright
% ...larger farms split up,
% \vspace{-5pt}
% \flushleft
%    \begin{figure}
% 		\includegraphics[scale=0.4]{Figures/initdist_example_1.png}
%    \end{figure}   
% \end{frame}

% \begin{frame}[noframenumbering]{Change in number of owners by size bins}

% \flushright
% ...smaller farms split up
% \vspace{-5pt}
% \flushleft

%    \begin{figure}
% 		\includegraphics[scale=0.4]{Figures/initdist_example_2.png}
%    \end{figure}   
% \end{frame}

% \begin{frame}[noframenumbering]{Change in number of owners by size bins}

% \flushright
% ...or mid-sized farms split up
% \vspace{-5pt}
% \flushleft

%    \begin{figure}
% 		\includegraphics[scale=0.4]{Figures/initdist_example_3.png}
%    \end{figure}   
% \end{frame}

% \begin{frame}[noframenumbering]{Change in number of owners by size bins}

% \begin{itemize}
%    \item We define land size bins according to the \emph{initial} distribution in each location
% \end{itemize}
% \vspace{-5pt}
%    \begin{figure}
% 		\includegraphics[scale=0.4]{Figures/initdist_example_0.png}
%    \end{figure}   
% \end{frame}

% \begin{frame}[noframenumbering]{Change in number of owners by size bins}

% \begin{itemize}
%    \item We define land size bins according to the \emph{initial} distribution in each location
% \end{itemize}
% \vspace{-5pt}

%    \begin{figure}
% 		\includegraphics[scale=0.4]{Figures/initdist_example_4.png}
%    \end{figure}   
% \end{frame}

% \begin{frame}[noframenumbering]{Change in number of owners by size bins}


% \begin{itemize}
%    \item and estimate the change in the number of owners in each \underline{fixed} area bin
% \end{itemize}

%    \begin{figure}
% 		\includegraphics[scale=0.4]{Figures/initdist_example_5.png}
%    \end{figure}   
% \end{frame}


% \begin{frame}[noframenumbering]{Change in number of owners by size bins}
% \vspace{-5pt}
% \begin{itemize}
% 	\item We define land size bins according to the \emph{initial} distribution in each location and estimate the change in the number of owners in each \underline{fixed} area bin
% \end{itemize}

%     \begin{figure}
% 		\includegraphics[scale=0.375]{Figures/initdist_example_7.png}
%     \end{figure}   
% \end{frame}

% \begin{frame}{Result 3: Sales translate into increase in number of \emph{small} farmers}\label{sli:quintiles}

% \vspace{-30pt}
% \begin{align*} 
%     NumOwners_{i, t}^{ j} = \gamma^j TempShocks_{i,t} + X'_{i,y}\xi^j + \eta_{i} + \kappa_{t} + \omega^j_{i,t},
% \end{align*}
% \vspace{-30pt}

% 	\begin{figure}
% 		\includegraphics[scale=0.38]{Figures/coefplot_lognum_owners_10quant.png}
% 	\end{figure}     

% \vspace{-30pt}
% \hyperlink{sli:quantilesq5}{\beamerbutton{q=5}} 
% \hyperlink{sli:quantilesq20}{\beamerbutton{q=20}} 	
	
% \end{frame}

% \begin{frame}{Sales by type of buyer}

% \begin{itemize}
% 		\smallskip
% 		\item Purchases by landless households or by farmers from neighboring municipalities?
% 		\smallskip
% 		\pause

% 		\item Sales records to build list of owners in $t-1$ at \textit{departamento} level. 
% 		\begin{itemize}
% 			\item Baldío allocation, land buyer, inheritance recipient
% 		\end{itemize}
% 		\smallskip
% 		\pause 

% 		\item Compare list of owners in $t-1$ with buyers at $t$:
% 	\end{itemize}

% \pause
% \bigskip 
% 	\centering
% 	\scalebox{0.8}{\input{Tables/reg-snr-ownermatch_dptolvl_3cols_small.tex}}

% \end{frame}


% % \begin{frame}{Mechanisms: Who buys land?}
% % 	\begin{itemize}
% % 		\item Use names on  allocations and transactions \textit{before} $t$.
% % 		\item Match by name with the list buyers at  $t$.
% % 		\item Classify  sales between those with a buyer who was a previous owner  and those with new buyers
% % 		%\vspace{0.5cm}
% % 		%\centering
% % 		%\scalebox{0.8}{\input{Tables/buyer_owner_match_rate}}
% % 	\end{itemize}
% % \end{frame}


% % \begin{frame}{Mechanisms: Who buys land?}
% % 	\begin{itemize}
% % 		\item Between 74\% and 82\% not in the list of previous owners in the departamento. 
% % 		\vspace{0.5cm}
% % 		\centering
% % 		\scalebox{0.6}{\input{Tables/reg_snr_owner_match}}
% % 	\end{itemize}
% % \end{frame}



% \begin{frame}{Summary of Results}
	
% 	\begin{itemize}
		
% 		\item Temperature shocks cause \emph{increases} in land sales and mortgages. 
% 		\medskip
		
% 		\item Temperature shocks cause \emph{reductions} in average farm size. 
% 		\medskip
		
% 		\item Effects concentrated in left tail of the farm size distribution.
% 		\medskip 

% 		\item Most land buyers are landless.
% 	\end{itemize}

% \bigskip 
% \begin{center}
% \textbf{What happens within the household?}
% \end{center}

% \end{frame}

% \begin{frame}{Supporting Evidence: Farmer's Decisions}
% 	\centering
% 	\scalebox{0.8}{\input{Tables/tab-elca-main.tex}}
% \end{frame}


% \begin{frame}{Robustness}\label{sli:robshockdef}
	
% 	\begin{itemize}	
		
		
% 		\medskip
% 		\item Alternative shock definitions:
		
% 		\begin{itemize}
			
% 			\smallskip
% 			\item Shocks as realizations outside $[5 pct; 95 pct]$ of temperature distribution\\
			
% 			\smallskip
% 			\item Shocks as realizations outside $[\mu -1.5 sd ; \mu + 1.5 sd ]$ of temp. distribution \\
			
% 			\smallskip
% 			\item Shorter time window to obtain temperature distribution 
			
% 		\end{itemize}
		
		
% 		\medskip
% 		\item Controls:
		
% 		\begin{itemize}
			
% 			\smallskip
% 			\item Department specific linear trends
			
% 			\smallskip
% 			\item Homicides
			
% 			\smallskip
% 			\item Displacement 
			
% 			\smallskip
% 			\item Cluster by municipality and department $\times$ year
			
% 		\end{itemize} 
		
		
% 		\medskip
% 		\item Alternative weather data:
		
% 		\begin{itemize}
			
% 			\smallskip
% 			\item SPEI
			
% 		\end{itemize} 
		
		
% 	\end{itemize}
	
	
% \end{frame}


% \begin{frame}{Summary of Results}
	
% 	\begin{itemize}
		
% 		\item Temperature shocks cause \emph{increases} in land sales and mortgages. 
% 		\medskip
		
% 		\item Temperature shocks cause \emph{reductions} in average farm size. 
% 		\medskip
		
% 		\item Effects concentrated in left tail of the farm size distribution. 
% 		\medskip 

% 		\item Most land buyers are landless.
% 		\medskip 

% 		\item At the household level:
% 		\begin{itemize}
% 			\smallskip
% 			\item Lower consumption and ownership of other assets
% 			\smallskip
% 			\item Increases in the likelihood of migration 
% 			\smallskip
% 			\item Decreases in likelihood of land ownership and increase the likelihood of owning smaller plots 
% 		\end{itemize}  	
% 	\end{itemize}
% \end{frame}



% \begin{frame}{Potential Explanations for Lack of Consolidation}\label{sli:mechanisms}

% %Section 5 develops a theoretical model that rationalizes our collection of results through the lack of access to insurance mechanisms. Farmers sell land and exit agriculture, in part, to smooth consumption when experiencing negative income shocks. In particular, small farmers are more likely to use these coping strategies because of their lower consumption levels, which implies higher marginal utility of consumption and greater gains from increasing present versus future consumption. Because the shock is temporary, the future value of becoming a farmer for landless households remains unaffected, but the price of land that landless households face still falls, because current farmers who are experiencing the negative shock are trying to sell their land to smooth consumption. Quantitatively, that can lead to an increase in the number of farmers and a decrease in average farm size after the shock.

% 	\begin{itemize}
% 	\item Frictions against consolidation due to non-contiguity of plots for sale {\scriptsize \color{gray}(e.g. Brooks and Lutz, 2016)}
% 	\pause

% 	\begin{itemize}
% 		\item[] $\rightarrow$ No evidence that this drives results  \hyperlink{sli:mech_contiguity_a}{\beamerbutton{Large Neighbor Prob.}}
% 	\end{itemize}
% 	\medskip 
% 	\pause 

% 	%Small farms might be more easily converted to residential or recreational purposes, which is more likely in the outskirts of cities due to urban expansion. We investigate this hypothesis by focusing on municipalities located farther than the median distance from main cities, where these alternative uses are more frequent.
	
% 	\item Small farms might be more likely converted to residential/recreational purposes
% 	\pause	

% 	\begin{itemize}
% 		\item[] $\rightarrow$ No evidence that this drives results \hyperlink{sli:mech_contiguity_a}{\beamerbutton{Distance to urban}}
% 	\end{itemize}
% 	\medskip
% 	\pause 

% 	\item Law 160 of 1994 forbids the accumulation of government-allocated land {\scriptsize \color{gray}(Arteaga, 2023)}
% 	\pause

% 	\begin{itemize}
% 		\item[] $\rightarrow$ No evidence that this drives results  \hyperlink{sli:mech_restrictions}{\beamerbutton{Land Market Restrictions}}
% 	\end{itemize}

% \end{itemize}

% \end{frame}

% \begin{frame}[noframenumbering]{Outline}
	
% 	\begin{itemize}
% 		{\Large
% 		\item[\textcolor{gray!50!white}{\textbullet}]\light{Motivation \& Contribution} \smallskip
% 		\item[\textcolor{gray!50!white}{\textbullet}]\light{Context \& Stylized Facts} \smallskip
% 		\item[\textcolor{gray!50!white}{\textbullet}]\light{Data} \smallskip
% 		\item[\textcolor{gray!50!white}{\textbullet}]\light{ Reduced Form Results} \smallskip
% 		\item Model \& Farm Size Dynamics \smallskip
% 		\item[\textcolor{gray!50!white}{\textbullet}]\light{Conclusion}}
% 	\end{itemize}

% \end{frame}
  

% \begin{frame}{Quantitative Model: Key elements}

% 	\begin{itemize}
% 		\item What are the key elements that we want in the model? \medskip
% 		\pause
% 			\begin{enumerate}
% 			\item \textbf{Heterogeneous agents with intertemporal choices} \\ \medskip $\Rightarrow$ model captures farm dynamics and consumption smoothing behavior \medskip
% 			\pause
% 			\item \textbf{Discrete occupational choices} \\ \medskip $\Rightarrow$ model captures exit of farmers and changes in avg farm size \medskip
% 			\pause
% 			\item \textbf{Aggregate shocks} \\ \medskip $\Rightarrow$ model captures the impact of shocks on the farm size distribution \medskip
% 			\end{enumerate}
% 			\pause
% 			\item Heterogeneous agent model with aggregate shocks + Discrete choices  \\ \medskip $\Rightarrow$ Computationally a hard model to solve, but enormous progress recently \\ \medskip $\Rightarrow$ Sequence-Space Jacobian (Auclert et al., 2022) + Discrete choice (Iskhakov et al., 2017)
% 	\end{itemize}

% \end{frame}


% \begin{frame}{Model}

% 	\begin{itemize}
% 		\item Time $t$ is discrete. Households are a triple $\omega_t=(s_t, w_t, \ell_t)$ and live forever. 
% 		\pause
% 		\item Agents choose land for the next period ($\ell_t\geq0$) and their occupation \{farmer, worker\}.
% 		\pause
% 		\item Production Technology is:
% 			\begin{align*}
% 				y_t = (Z_t s_t)^{1-\gamma}(l^\alpha_t k^{1-\alpha}_t)^\gamma,
% 			\end{align*}
% 			\item[] where $Z_t$ is aggregate productivity, $s_t$ idiosyncratic.
% 			\pause

% 		\item Rental rate $r$ of $k$ exogenous.
% 		\pause

% 		\item Workers are landless ($\ell_t = 0$), and earn a wage $w_t$ \underline{unaffected by $Z_t$}.
% 		\pause

% 		\item Farmers earn no wage ($w_t = 0$), and make profits $\pi(Z_t, s_t, \ell_t)$

% 	\end{itemize}

% \end{frame}

% \begin{frame}{Timing}

% 	\begin{center}
% 		\includegraphics[scale=0.2]{Figures/model_timing.png} 
% 	\end{center}

% \end{frame}

% \begin{frame}{Optimization Problem}

% 	\begin{itemize}
% 		\item Recursively, the household's problem is:  
% 		\begin{equation}
% 			V\left( \omega_t \right)=\left(1-\delta\right)\max\left\{ V_{F}\left( \omega_t \right)+\varepsilon_{F},V_{W}\left( \omega_t \right)+\epsilon_{W}\right\} +\delta V_{W}\left( \omega_t \right)\label{eq:V}
% 		\end{equation}

% 		\pause

% 	\item where value of becoming a farmer is: \begin{align}
% 		V_{F}\left( \omega_t \right) & =\max_{c_{t},\ell_{t+1}}\left\{ u\left(c_{t}\right)+\beta \mathbb{E}_{s}\left( \mathbb{E} _{\varepsilon}\left(V\left( \omega_{t+1} \right)\right)|s_{t}\right)\right\} \label{eq:V_F}\\
% 		\text{s.t. }c_{t} & =\pi\left(Z_{t},\omega_t \right) + p_{t}\left(\ell_{t}-\ell_{t+1}\right),\nonumber 
% 		\end{align}

% 		\pause

% 	\item and value of becoming a worker is: \begin{align}
% 		V_{W}\left( \omega_t \right) & =\max_{c_{t}}\left\{ u\left(c_{t}\right)+\beta V_{0}\left( \omega_t \right)\right\} \label{eq:V_W}\\
% 		\text{s.t. }c_{t} & =w_{t}+ p_{t}\ell_{t}\nonumber 
% 		\end{align}
	
% 	\item where $\varepsilon_F$, $\epsilon_W$ are occupation taste shocks.
% 	\end{itemize}
% \end{frame}

% \begin{frame}{Distributional Assumptions}

% 	\begin{itemize}
% 		\item Farming productivity $s_{t}$ follows a stochastic process $\log s_{t+1}=\rho\log s_{t}+\sigma\epsilon_{t+1}$, where $\epsilon_{t}\sim\mathcal{N}\left(0,1\right)$.
% 		%\epsilon_ is a productivity shock, $\rho\in(0,1)$ is the persistence, and $\sigma>0$ is the volatility of $\epsilon_{t}$. 
		
% 		\pause 
% 		\item The wage distribution of workers comes, given by $G_{w}\sim\log\mathcal{N}\left(\mu_{w},\sigma_{w}\right)$
		
% 		\pause 
% 		\item  Taste shocks $\varepsilon_{F}$ and $\epsilon_{W}$ are drawn independently from an (EVT1) distribution with dispersion parameter $\kappa$.
% 		\medskip 

% 		\pause
% 		\item Allows to compute the probability of $\omega_t$ choosing farming:
		
% 		\begin{equation}
% 			\mu\left( \omega_t \right)=\dfrac{\exp\left(\dfrac{1}{\kappa}V_{F}\left( \omega_t \right)\right)}{\exp\left(\dfrac{1}{\kappa} \mathbb{E}_{\varepsilon}\left(V\left( \omega_t \right)\right) \right)},\label{eq:pstay}
% 		\end{equation}

% 		\item[] where $\mathbb{E}_{\varepsilon}\left(V\left( \omega_t \right)\right)=\kappa\log\left(\exp\left(\dfrac{1}{\kappa}V_{F}\left( \omega_t \right)\right)+\exp\left(\dfrac{1}{\kappa}V_{W}\left( \omega_t \right)\right)\right)$
% 	\end{itemize}


% \end{frame}

% \begin{frame}{Stationary Equilibrium}

% 	\begin{itemize}
% 		\item Occupational choices define the share of households in agriculture  $\rightarrow$ long run defines a stationary \emph{distribution} of workers and farmers $\{G^{*}\}$.
% 		\smallskip
% 		\pause 

% 		\item Price of land $p_t$ ensures land markets clear:
% 		\begin{equation}
% 			L_{t}^{D}=\int \ell^{*}\left( \omega_t \right)h_{F}\left( \omega_t \right)Nd\omega_t = L
% 		\end{equation}
% 		\pause 

% 		\item Competitive equilibrium are vectors $\{\ell^{*}$,$c^{*}\}$, a land price $\{p^{*}\}$, and an invariant distribution $\{G^{*}\}$ such that:
% 		\begin{itemize}
% 			\item Land and consumption choices are optimal
% 			\item Individual occupational choices are optimal 
% 			\item Land markets clear
% 			\item Stationary distribution holds
% 		\end{itemize}
% 	\end{itemize}

% \end{frame}

% \begin{frame}{Calibration}

% 	\begin{table}
% 	\resizebox{0.9\textwidth}{!}{ % Resize table to fit slide width
% 	\begin{tabular}{lrllr}
% 		\hline 
% 		{ Symbol} & { Value} & { Description} & { Target/Source} & { Target Value}\tabularnewline
% 		\hline 
% 		\multicolumn{1}{l}{{$\gamma$}} & { 0.47} & { Share of inputs} & { Avila and Evenson (2010)} & { -}\tabularnewline
% 		{$\alpha$} & { 0.22} & { Share of land} & { Avila and Evenson (2010)} & { -}\tabularnewline
% 		{$\beta$} & { 0.96} & { Discount rate} & { Greenwod et al. (2010)} & { -}\tabularnewline
% 		{$\kappa$} & { 4} & { Occupational choice elasticity} & { Migration Literature} & { -}\tabularnewline
% 		{$\rho$} & { 0.75} & { Serial correlation in yields} & { Regression of yields} & { -}\tabularnewline
% 		{$N$} & { 0.17} & { Total mass of agents} & { Average Farm Size} & { 5.98}\tabularnewline
% 		{$\sigma$} & { 2.2} & { Volatility of shocks} & { S.d. of farm size} & { 15.50}\tabularnewline
% 		{$\delta$} & { 0.05} & { Exogenous exit probability} & { Share of large farms who exit} & { 0.05}\tabularnewline
% 		{$V_{0}^{F}$} & { -51.1} & { Value of becoming a worker} & { Share of small farms who exit} & { 0.08}\tabularnewline
% 		{$V_{0}^{W}$} & { 20.3} & { Value of staying as a worker} & { Share of workers in agriculture} & { 0.15}\tabularnewline
% 		{$\mu_{w}$} & { -2.37} & { Average of log wage} & { Mean farm size of new entrants} & { 1.32}\tabularnewline
% 		{$\sigma_{w}$} & { 1.12} & { S.d. of log wage} & { S.d. of farm size of new entrants} & { 2.39}\tabularnewline
% 		{$Z_{ss}$} & { 1} & { SS aggregate productivity} & { Normalization} & { -}\tabularnewline
% 		{$L^{S}$} & { 1} & { Total supply of land} & { Normalization} & { -}\tabularnewline
% 		\hline 
% 		\end{tabular}
% 		}%
% 		\end{table}
% \end{frame}

% \begin{frame}{Farm Size Dynamics}

% 	\begin{figure}
% 		\caption{Farm Size Distribution and Misallocation in the Stationary Equilibrium\protect\label{fig:farmsize}}
% 		\begin{center}
% 		\subfloat[Farm Size Distribution]{\centering\includegraphics[scale=0.35]{Figures/fig_fsize_optimal_actual}}
% 		\subfloat[Misallocation]{\centering\includegraphics[scale=0.35]{Figures/fig_mpl_vs_farm_productivity}}
% 		\par\end{center}
% 	\end{figure}
		
% \end{frame}

% \begin{frame}{Weather Shocks and Transition Dynamics}

% 	\begin{itemize}
% 		\item Simulate a 25\% drop in aggregate productivity ($Z_t$)
% 		\item Shock is temporary (low corr between $(Z_t, Z_{t+1}$))
% 	\end{itemize}
% 	\vspace{-10pt}
% 	\begin{figure}
% 	\setlength{\abovecaptionskip}{0pt}  % Reduce space above caption
%     \setlength{\belowcaptionskip}{0pt}  % Reduce space below caption
% 	%\caption{Dynamic Effects of a Weather Shock\protect\label{fig:farmsize_dynamics_ii}}
% 	\begin{centering}
% 		\subfloat[Avg. Farm Size]{
% 		\centering{}\includegraphics[scale=0.32]{Figures/fig_dyn_fsize}}\subfloat[Avg. Skill]{
% 		\centering{}\includegraphics[scale=0.32]{Figures/fig_dyn_skill}}\subfloat[Optimal to Actual Ratio]{
% 		\centering{}\includegraphics[scale=0.32]{Figures/fig_dyn_optimal}}
% 	\par\end{centering}
% 	\end{figure}

% \begin{itemize}
% 	\item Average farm size falls by 0.8\% and takes 30+ years to recover
% 	\item Shock reduces average skill size and exacerbates misallocation
% \end{itemize}

% \end{frame}

% \begin{frame}{Weather Shocks and Transition Dynamics}

% 	\begin{figure}
% 		\setlength{\abovecaptionskip}{0pt}  % Reduce space above caption
%         \setlength{\belowcaptionskip}{0pt}  % Reduce space below caption
% 		%\caption{Dynamic Effects of a Weather Shock\protect\label{fig:farmsize_dynamics}}
% 		\begin{centering}
% 		\subfloat[Land Price]{
% 		\centering{}\includegraphics[scale=0.32]{Figures/fig_dyn_price}}\subfloat[Total Ag-Output]{
% 		\centering{}\includegraphics[scale=0.32]{Figures/fig_dyn_output}}\subfloat[Farmers' Consumption]{
% 		\centering{}\includegraphics[scale=0.32]{Figures/fig_dyn_consumption}}
% 		\par\end{centering}
% 	\end{figure}		

% 	\begin{itemize} 
% 		\item Price and output drop. 
% 		\item Farmers that stay have larger reductions in consumption relative to those who exit.
% 	\end{itemize}
	
% \end{frame}

% \begin{frame}{Weather Shocks and Transition Dynamics}

% 	\begin{figure}[h!]
% 		\setlength{\abovecaptionskip}{0pt}  % Reduce space above caption
%         \setlength{\belowcaptionskip}{0pt}  % Reduce space below caption
% 		\caption{Impact of a Weather Shock on Exit Probability at $t=0$\protect\label{apx_fig:exit_probability}}
% 		\begin{center}
% 		\subfloat[Farm Size]{\centering\includegraphics[scale=0.3]{Figures/fig_exit_vs_land}}
% 		\subfloat[Farm Productivity]{\centering\includegraphics[scale=0.3]{Figures/fig_exit_vs_z}}
% 		\end{center}
% 	\end{figure}

% 	\begin{itemize} 
% 		\item Smaller, less productive farmers more likely to exit. 
% 	\end{itemize}

% \end{frame}



% % \begin{frame}[noframenumbering]{Outline}
	
% % 	\begin{itemize}
% % 		{\Large
% % 		\item[\textcolor{gray!50!white}{\textbullet}]\light{Motivation \& Contribution} \smallskip
% % 		\item[\textcolor{gray!50!white}{\textbullet}]\light{Context \& Stylized Facts} \smallskip
% % 		\item[\textcolor{gray!50!white}{\textbullet}]\light{Data} \smallskip
% % 		\item[\textcolor{gray!50!white}{\textbullet}]\light{Reduced Form Results} \smallskip
% % 		\item[\textcolor{gray!50!white}{\textbullet}]\light{Model \& Farm Size Dynamics} \smallskip
% % 		\item Conclusion}
% % 	\end{itemize}

% % \end{frame}

% \begin{frame}{Discussion}

% 	\begin{itemize}
% 		\item Negative, sectoral shock induces farmers to sell their land to smooth consumption.
% 		\bigskip 
% 		\pause 

% 		\item Land price drops and becomes more attractive. For workers, long-run value of becoming a farmer not strongly affected because shock is temporary.
% 		\smallskip
% 		\pause
% 		\begin{itemize}
% 			\item Attracts landless households into farming in larger proportion than exit rates.
% 			\smallskip
% 			\pause
% 			\item Main mechanisms are i) asymmetric shock between sectors, ii) Low persistence of shock across time.
% 		\end{itemize}
% 		\bigskip 
% 		\pause 

% 		\item Shock exacerbates misallocation, but this is present even in stationary equilibrium.
		
% 	\end{itemize}
% \end{frame}

% \begin{frame}{Conclusion}

% 		\begin{itemize}
% 			\item Uninsured weather shocks lead to increases in land sales and decreases in average farm size. 
% 			\medskip
% 			\pause 

% 			\item Evidence suggests land sales play important role as a buffer. 
% 			\medskip
% 			\pause

% 			\item GE effects lead to higher rates of small farms after shocks -- contrary to hypothesis of shocks driving land consolidation.
% 			\begin{itemize}
% 				\item Important implications for long-run productivity.
% 			\end{itemize} 
% 			\medskip
% 			\pause 

% 			\item Next steps: use theoretical model to evaluate two sets of counterfactuals 
% 			\smallskip
% 		   \begin{itemize}
% 			 \item Increasing rates of agricultural insurance adoption. \smallskip
% 			 \item Higher shock prevalence/intensity based on climate change projections.
% 		   \end{itemize}
% 		\end{itemize}
% \pause

% \flushright
% \textbf{Thanks!}

% \textbf{jgarteagav@gmail.com}
% \end{frame}

% % -------------------------------------------------------------------
% % -------------------------------------------------------------------

% % %%%%%%%%%%%%%%%%%%%%%%%%%%%%%%%%%%%%%%%%%%%%%%%
% % %                                             %
% % %                Appendix                     %
% % %                                             %
% % %%%%%%%%%%%%%%%%%%%%%%%%%%%%%%%%%%%%%%%%%%%%%%%

% % -------------------------------------------------------------------
% % -------------------------------------------------------------------

% \backupbegin

% \appendix

% \begin{frame}

% 	\centering \Large \textbf{Appendix}
	
% \end{frame} 

% \begin{frame}{Appendix: Stylized facts}\label{sli:appxmotivatingtab}
	
% 	\centering
% 	\scalebox{1}{\input{Tables/motivating_regs_table.tex}}

% 	\flushright   
% 	\hyperlink{sli:stylized_ii}{\beamerbutton{Back}}

% \end{frame}


% \begin{frame}{Appendix: Share of farms below region-specific threshold (UAF)}\label{sli:apxfragment}
	
% 	\begin{figure}
% 		\centering
% 		\includegraphics[scale=0.40]{Figures/scatter-fragmentacion-uaf.png}
% 	\end{figure} 
% \vspace{-25pt}    
% 	\hyperlink{sli:context}{\beamerbutton{Back}}
	
% \end{frame}


% \begin{frame}{Appendix: Land Sales}\label{sli:transtypetime}

% \begin{figure}
% 	\centering
% 	\begin{subfigure}{0.35\textwidth}
% 		\centering
% 		\includegraphics[width=0.75\linewidth]{Figures/fig_salesmap.png}
% 		\caption{Sales as fraction of Allocations}
% 	\end{subfigure}%
% 	\begin{subfigure}{0.5\textwidth}
% 		\centering
% 		\includegraphics[width=\linewidth]{Figures/fig_transactions_2000_2011.png}
% 		\caption{Number of Transactions}
% 	\end{subfigure}    
% \end{figure}    
% 	\hyperlink{sli:data}{\beamerbutton{Back}}  

% \end{frame} 


% \begin{frame}{Appendix: Temperature Shocks, 2000}\label{sli:apxshocksmap}
		
% 	\begin{figure}
% 		\centering
% 		\includegraphics[scale=0.55]{Figures/fig_mapshocks2000_NdR_forslide.png}
% 	\end{figure}     
	
% 	\hyperlink{sli:defshocks}{\beamerbutton{Back}}
	
% \end{frame}


% \begin{frame}{Appendix: Temperature Shocks, 2010}
		
% 	\begin{figure}
% 		\centering
% 		\includegraphics[scale=0.55]{Figures/fig_mapshocks2010_NdR_forslide.png}
% 	\end{figure}     
	
% 	\hyperlink{sli:defshocks}{\beamerbutton{Back}}
	
% \end{frame}

% \begin{frame}{Appendix: Descriptive Statistics}\label{sli:apxdescriptives}
% 	\centering
% 	\scalebox{0.34}{
% 		\input{Tables/tab-descriptives}}
% 	\medskip
	
% \flushleft	
% \vspace{-25pt}
% \hyperlink{sli:empstrategy}{\beamerbutton{Back}}
	
% \end{frame}

% \begin{frame}{Robustness}\label{sli:robustness}

% 	\begin{itemize}	
% 		\item  Shocks as realizations outside $[10 pct, 90 pct]$ of temperature distribution. 
	
% 		\medskip
% 		\item  Shocks as realizations outside $[5 pct; 95 pct]$ of temperature distribution. 
	
% 		\medskip
% 		\item Shocks as realizations outside $[\mu -1.5 sd ; \mu +1.5 sd ]$ of temperature distribution. 
	
% 		\medskip
% 		\item Shocks as realizations outside $[\mu -2 sd ; \mu + 2 sd ]$ of temperature distribution. 
		
% 		\medskip
% 		\item Shocks defined as days above/below country level temperature thresholds  {\scriptsize \color{gray} (Aguilar-Gomez, et. al., 2022)}.
				
% 		\medskip
% 		\item Exclusion of additional controls. 
% 	\end{itemize}

% 	\hyperlink{sli:defshocks}{\beamerbutton{Back}}

% \end{frame}


% \begin{frame}{Appendix: Effect on Land Sales, Lags of Shocks}\label{sli:apxeffsalelags}
	
% 	\begin{figure}
% 		\centering
% 		\includegraphics[width=0.7\textwidth]{Figures/coefplot-snr-lags.png}
% 	\end{figure}     
% 	\vspace{-30pt}
% 	\hyperlink{sli:sales}{\beamerbutton{Back}}
	
% \end{frame}

% \begin{frame}{Appendix: Effect on Number of Owners, Lags of Shocks}\label{sli:apxeffownerslags}
		
% 	\begin{figure}
% 		\centering
% 		\includegraphics[width=0.7\textwidth]{Figures/fig_eff_numowners_lags.png}
% 	\end{figure}     
% 	\vspace{-30pt}
% 	\hyperlink{sli:effsize}{\beamerbutton{Back}}
	
% \end{frame}


% \begin{frame}{Appendix: Effect on Farm Size, Lags of Shocks}\label{sli:apxeffsizelags}
		
% 	\begin{figure}
% 		\centering
% 		\includegraphics[width=0.7\textwidth]{Figures/fig_eff_size_lags.png}
% 	\end{figure}     
% 	\vspace{-30pt}
% 	\hyperlink{sli:effsize}{\beamerbutton{Back}}
	
% \end{frame}


% \begin{frame}{Appendix: Sales translate into increase in number of \emph{small} farmers}\label{sli:quantilesq5}
		
% 	\begin{figure}
% 		\centering
% 		\includegraphics[scale=0.3]{Figures/coefplot_lognum_owners_5quant.png}
% 	\end{figure}     
	
% 	\hyperlink{sli:quintiles}{\beamerbutton{Back}}
	
% \end{frame}

% \begin{frame}{Appendix: Sales translate into increase in number of \emph{small} farmers}\label{sli:quantilesq20}
		
% 	\begin{figure}
% 		\centering
% 		\includegraphics[scale=0.3]{Figures/coefplot_lognum_owners_20quant.png}
% 	\end{figure}     
	
% 	\hyperlink{sli:quintiles}{\beamerbutton{Back}}
	
% \end{frame}

% \begin{frame}{Mechanisms: Land Market Restrictinos}\label{sli:mech_restrictions}

% 	\begin{itemize}
		
% 		\item Law 160 of 1994 forbids accumulation of land of the public allocation program.
% 		\begin{itemize}
% 			\item Large owners cannot accumulate land. 
% 			\item If law explains results, effects concentrated in municipalities with more land allocation. 
% 		\end{itemize}
% 	\end{itemize}	
		
% 	\centering
	
% 	\scalebox{0.8}{\input{Tables/reg-catastro-alloc.tex}}

% 	\hyperlink{sli:mechanisms}{\beamerbutton{Back}}
	
% \end{frame}


% \begin{frame}{Mechanisms: Demand Side}\label{sli:mech_contiguity_a}
	
% 	\centering
% 	\scalebox{0.6}{\input{Tables/tab-catastro-heter.tex}}

% 	\hyperlink{sli:mechanisms}{\beamerbutton{Back}}
	
% \end{frame}

% % \begin{frame}{Appendix: Model Structure}\label{sli:modelstructure1}
   
% %     \begin{itemize}
% %         \item Endowments, Occupation, Technology: 
% %         \smallskip
% %         \begin{itemize}
% %             \item Agents are endowed with some land ($l_0$) or wealth ($m_0$)
% %             \item Occupation is a discrete choice: agents who choose to hold land can't be wage workers
% %             \begin{itemize}
% %                 \item[-] Decision depends on which occupation yields highest utility 
% %             \end{itemize}
% %             \item All agents who choose to hold land have same skill and use the same CRS technology: $$y(l)=al$$
% %             \smallskip
    
% %             \item A `shock' is a change in TFP: $a = a_L$ in $t_1$, $a = a_H$ in $t_2$; $a_L < a_H$
% %             \smallskip
    
% %             \item Agents who choose to work in non-farm sector earn a fixed wage $w$
% %             \smallskip
    
% %             \item Both farmers and workers can hold `wealth' asset which has a fixed, exogenous, return each period $r_1$, $r_2$ 
% %         \end{itemize}
% %     \end{itemize}

% %     \hyperlink{sli:rationalizing}{\beamerbutton{Back}}

% % \end{frame}


% % \begin{frame}{Appendix: Model Structure}
       
% % 	\begin{itemize}
% % 		\item Timing: 
% % 		\vspace{10pt}
% % 		\begin{itemize}
% % 			\item $t_0$: Agents are endowed with asset ($m_0$), or land ($l_0$)
% % 			\item $t_1$: Agents decide how much land and asset to hold $\{m^*_1, l^*_1\}$, and consumption ($c_1$)
% % 			\item $t_2$: Agents consume ($c_2$) according to their asset and land choices in $t_1$.
% % 		\end{itemize}
% % 	\end{itemize}
% % 	\bigskip
    
% % 	An equilibrium is a land price $p_1$ and a vector of land and wealth demands $\{l^*,m^*\}$ such that i) each agent is maximizing utility and ii) land markets clear
    
% %     \hyperlink{sli:rationalizing}{\beamerbutton{Back}}

% % \end{frame}        
    
% %     \begin{frame}{Appendix: Model Structure}
    
% %     \begin{itemize}
% %         \item Preferences:
% %         \begin{align*}
% %             U=\log\left(c_{1}-c_{S}\right)+\log\left(c_{2}-c_{S}\right)
% %         \end{align*}
% %         \item Budget constraints:
% %         \begin{itemize}
% %             \item For farmers: 
% %             \begin{align*}
% %                 a_L(l_0 + l_1) - p_1l_1 + r_1m_0 - m_1 &= c_1 \\
% %                 a_H(l_0 + l_1) + r_2m1 &= c_2
% %             \end{align*}
    
% %             \item For workers:
% %             \begin{align*}
% %                 w + p_1l_0 + r_1m_0 - m_1 &= c_1 \\
% %                 w + r_2m_1 &= c_2
% %             \end{align*}
% %         \end{itemize}
        
% %         \begin{itemize}
% %             \item Farmers choose how much land to buy or sell $(l_1)$, and how much wealth to keep for next period $(m_1)$
% %             \item Workers sell all of their endowed land $(-l_0)$, choose wealth $(m_1)$ and earn wage $(w)$
% %         \end{itemize}
        
% %     \end{itemize}
    
% %     \hyperlink{sli:rationalizing}{\beamerbutton{Back}}

% % \end{frame}
    
% %     \begin{frame}{Appendix: Individual solution}
    
% %         Solution to the individual maximization problem yields:
    
% %         \begin{itemize}
% %             \item For Farmers:
% %             \begin{align*}
% %                 l_{1,F}^{*} & =\dfrac{\left(2a_{L}-p_{1}\right)}{2\left(p_{1}-a_{L}\right)}\left(l_{0}+\frac{r_1}{a_{L}}m_{0}\right)+\dfrac{\left(p_{1}-a_{L}-a_{H}\right)}{2a_{H}\left(p_{1}-a_{L}\right)}c_{S}\\
% %                 m_{1,F}^{*} & =0\\
% %                 U_{F}^{*} & =\log\left(a_{L}l_{0}+r_{1}m_{0}-\left(p_{1}-a_{L}\right)l_{1}^{*}-c_{S}\right)+\log\left(a_{H}\left(l_{0}+l_{1}^{*}\right)-c_{S}\right)
% %             \end{align*}
% %             \smallskip
    
% %             \item For Workers:
% %             \begin{align*}
% %                 l_{1,W}^{*} & =-l_{0}\\
% %                 m_{1,W}^{*} & =\dfrac{1}{1+r_{2}}\left\{ c_{S}\left(1-r_{2}\right)-w\left(1-r_{2}\right)+r_{2}r_{1}m_{0}+r_{2}p_{1}l_{0}\right\} \\
% %                 U_{W}^{*} & =\log\left(a_{L}l_{0}+r_{1}m_{0}+w-m_{1}^{*}-c_{S}\right)+\log\left(w+r_{2}m_{1}^{*}-c_{S}\right)
% %             \end{align*}
% %         \end{itemize}
% %     \hyperlink{sli:rationalizing}{\beamerbutton{Back}}
 
% % \end{frame}
    
% %     \begin{frame}{Appendix: General Equilibrium}
    
% %     \begin{itemize}
% %         \item For any given land price $(p_1)$ each agent:
% %         \begin{itemize}
% %             \item Computes $\{l_{1,F}^{*}, m_{1,F}^{*}, U_{F}^{*}\}$; $\{l_{1,W}^{*}, m_{1,W}^{*}, U_{W}^{*}\}$
% %             \item Chooses to be a farmer if $U_{F}^{*} \geq  U_{W}^{*}$
% %             \item demands $l^*_1$ at price $p_1$.
% %         \end{itemize}
        
% %         \item In GE: 
% %         \begin{itemize}
% %             \item Aggregate land demand has to equal aggregate land supply
% %             \begin{align*}
% %                 \int_{\omega\in\Omega_{F}^{\ell}}l^{*}\left(\omega\right)dF\left(\omega\right)=\int_{\omega\in\Omega_{W}^{\ell}}l^{*}\left(\omega\right)dF\left(\omega\right)
% %             \end{align*}
% %         \end{itemize}
% %     \end{itemize}
    
% %     \hyperlink{sli:rationalizing}{\beamerbutton{Back}}

% % \end{frame}

% % \begin{frame}{Appendix: Model Results}\label{sli:modelresults1}

% %     \begin{itemize}
% %         \item $\downarrow$ productivity $\Rightarrow \downarrow$ average farm size,$\uparrow$ number of farmers
% %     \end{itemize}

% %     \begin{figure}
% % 		\centering
% %     \includegraphics[scale=0.65]{Figures/fig_model8_baseline.png}
% %     \end{figure}

% % \vspace{-30pt}
% % \hyperlink{sli:rationalizing}{\beamerbutton{Back}}

% % \end{frame}

% % \begin{frame}{Appendix: Model Results}\label{sli:modelresults2}

% %     \begin{itemize}
% %         \item $\downarrow$ productivity $\Rightarrow$ results come from left tail
% %     \end{itemize}

% %     \begin{figure}
% % 		\centering
% %     \includegraphics[scale=0.65]{Figures/fig_model8_change_numfarmers_prct.png}
% %     \end{figure}

% % \vspace{-30pt}
% % \hyperlink{sli:rationalizing}{\beamerbutton{Back}}

% % \end{frame}

% % \begin{frame}{Appendix: Model Results}\label{sli:modelresults3}

% %     \begin{itemize}
% %         \item $\downarrow$ productivity $\Rightarrow$ farmers leave agriculture
% %     \end{itemize}

% %     \begin{figure}
% % 		\centering
% %     \includegraphics[scale=0.65]{Figures/fig_model8_change_ownership_prct.png}
% %     \end{figure}

% % \vspace{-30pt}
% % \hyperlink{sli:rationalizing}{\beamerbutton{Back}}

% % \end{frame}

% % \begin{frame}{Appendix: Model Results}\label{sli:modelresults4}

% %     \begin{itemize}
% %         \item $\downarrow$ productivity $\Rightarrow$ Results stronger if region is poorer/less connected to markets:
% %     \end{itemize}

% %     \begin{figure}
% % 		\centering
% %         \includegraphics[scale=0.65]{Figures/fig_model8_poor_vs_rich.png}
% % 	\end{figure}     

% % \vspace{-30pt}
% % \hyperlink{sli:rationalizing}{\beamerbutton{Back}}

% % \end{frame}
% \backupend
% \end{document}  
  

