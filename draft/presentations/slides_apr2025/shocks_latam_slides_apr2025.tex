\documentclass[aspectratio = 169]{beamer}
%\documentclass[handout, notes = show]{beamer} %Para mostrar notas. 

%%%%%%%%%%%%%%%%%%%%%%%%%%%%%%%%%%%%%%%%%%%%%%%
%                                             %
%             Load Packages                   %
%                                             %
%%%%%%%%%%%%%%%%%%%%%%%%%%%%%%%%%%%%%%%%%%%%%%%

\usetheme{default}
\usepackage{amsmath} % NdR: for \begin{align} to work.
\usepackage{bbm} % NdR: needed indicator function symbol
\usepackage[utf8]{inputenc} % NdR: tildes en español. 
\usepackage{adjustbox} % NdR: to insert big tables into presentations. From .tex of Margarita Gáfaro.
\setbeamertemplate{navigation symbols}{} 
\setbeamertemplate{itemize items}[circle]
\setbeamertemplate{itemize subitem}{$-$}
\usefonttheme[onlymath]{serif}

\usepackage{graphicx} % NdR: varios de gráficos. 
\usepackage{multirow} % NdR: para multirows in tables.
\usepackage{comment}  % NdR: para comentarios de secciones enteras de texto.
\usepackage{booktabs} % NdR: para el comando \cmidrule que facilita hacer lineas horizontales en las tablas sin que se toquen.
\usepackage{threeparttable} %NdR: for tables (and table notes).
\usepackage{xcolor} %NdR: para definir colores y colorear ecuaciones. 
\usepackage{caption}
\usepackage{subcaption}
	
	
%%%%%%%%%%%%%%%%%%%%%%%%%%%%%%%%%%%%%%%%%%%%%%%
%                                             %
%              Slides Settings                %
%                                             %
%%%%%%%%%%%%%%%%%%%%%%%%%%%%%%%%%%%%%%%%%%%%%%%	
	
%--- NdR: covered parts of slide displayed transparently.
% \setbeamercovered{transparent} 
%\setbeamercovered{invisible}

% %--- NdR: definine blue	
% \definecolor{azulndr}{RGB}{52,113,206} 
	
% %--- NdR: set color of slides
% \setbeamercolor*{title}{bg=white,fg=azulndr}
% \setbeamercolor*{item}{fg= azulndr}
% \setbeamercolor{frametitle}{fg=azulndr}
	
% %--- NdR: Set color of slide buttons
% \setbeamercolor{button}{fg= white, bg=  azulndr}
% \renewcommand{\beamerreturnbutton}[1]{%
% 	\begingroup% keep color changes local
% 	\setbeamercolor{button}{fg= white , bg= azulndr}%
% 	\beamerbutton{\insertreturnsymbol#1}% original definition
% 	\endgroup
% }
	
%--- NdR : Change the margins of one of the pages:
%  from http://stackoverflow.com/questions/1670463/latex-change-margins-of-only-a-few-pages©
%  To use: \begin{changemargin}{-1cm}{-1cm} don't forget to \end{changemargin}
\newenvironment{changemargin}[2]{%
	\begin{list}{}{%
			\setlength{\topsep}{0pt}%©
			\setlength{\leftmargin}{#1}%
			\setlength{\rightmargin}{#2}%
			\setlength{\listparindent}{\parindent}%
			\setlength{\itemindent}{\parindent}%
			\setlength{\parsep}{\parskip}%
		}%
	\item[]}{\end{list}}
	
%--- NdR: to change margin of all the frames:
\setbeamersize{text margin left= 0.2in, text margin right= 0.2in}

	
% --- NdR: Put number to slide
	\addtobeamertemplate{navigation symbols}{}{%
		\usebeamerfont{footline}%
		\usebeamercolor[fg]{footline}%
		\hspace{1em}%
		\insertframenumber/\inserttotalframenumber}
	\addtocounter{page}{-1}

% --- Restart counter for appendix slides
\newcommand{\backupbegin}{
   \newcounter{framenumberappendix}
   \setcounter{framenumberappendix}{\value{framenumber}}
}
\newcommand{\backupend}{
   \addtocounter{framenumberappendix}{-\value{framenumber}}
   \addtocounter{framenumber}{\value{framenumberappendix}} 
}

% To shade gray: deemphasize text
\newcommand{\light}[1]{\textcolor{gray!50!white}{#1}}

\newcommand{\sym}[1]{\ifmmode^{#1}\else\(^{#1}\)\fi}

%%%%%%%%%%%%%%%%%%%%%%%%%%%%%%%%%%%%%%%%%%%%%%%
%                                             %
%              Title Page                     %
%                                             %
%%%%%%%%%%%%%%%%%%%%%%%%%%%%%%%%%%%%%%%%%%%%%%%

 \vspace*{-0.5cm}

\title[]{Shocks in Latin American Household Surveys: \\ Incidence and Consequences}

 \vspace*{0.1cm}

%  \author[Luis Martinez]{}
%  \institute[Emory]{\small{\begin{tabular}{c c c c c}
% %  \textbf{Julián Arteaga} & \textbf{Nicolás de Roux} & \textbf{Magarita Gáfaro} & \textbf{Ana M. Ibáñez} & \textbf{Heitor Pellegrina} \\
% %  World Bank & Los Andes & BanRep & IDB & Notre Dame\\ \\
%  \end{tabular}}}
 
\date{
	% \small{\small Universidad del Rosario \\ 
	
	% \vspace{0.5cm}
	April 2025}

%%%%%%%%%%%%%%%%%%%%%%%%%%%%%%%%%%%%%%%%%%%%%%%
%                                             %
%                  Slides                     %
%                                             %
%%%%%%%%%%%%%%%%%%%%%%%%%%%%%%%%%%%%%%%%%%%%%%%

\begin{document}

\begin{frame}
  \titlepage
\end{frame}
\begin{frame}{Research Question}

	\begin{itemize}
  
	  \item \textbf{Big picture:} How do households cope with different types of shocks?
		\begin{itemize}
		  \item What type of policies can foster mitigation?
		\end{itemize}
		\bigskip
  
	  \item \textbf{Literature:}
		\begin{itemize}
  
		  \item Large literature on impact of temporary vs.\ permanent shocks...
			\begin{itemize}
			  \item \textcolor{gray}{Deaton (1991); Paxson (1992); Townsend (1994) \dots}
			\end{itemize}
			\smallskip
  
		  \item Consumption smoothing, income smoothing, poverty traps...
			\begin{itemize}
			  \item \textcolor{gray}{Rosenzweig \& Wolpin (1993); Morduch (1995); Carter \& Zimmerman (2001)}
			\end{itemize}
			\smallskip
  
		  \item Impact of weather shocks / natural disasters:
			\begin{itemize}
			  \item \textcolor{gray}{Ibáñez et al.\ (2024); Deryugina et al.\ (2018); Kocornik-Mina (2020) \dots}
			\end{itemize}
			\smallskip
  
		  \item Impact of health shocks:
			\begin{itemize}
			  \item \textcolor{gray}{Fadlon \& Nielsen (2021); Mohanan (2013); Garcia-Gomez et al.\ (2013)}
			\end{itemize}
			\smallskip
  
		  \item Policy tools:
			\begin{itemize}
			  \item \textcolor{gray}{de Janvry et al. (2016): \emph{Progresa}}; \textcolor{gray}{Pople et al. (2021): Anticipatory transfers (drought)}; \textcolor{gray}{Carter \& Jansen (2019): Insurance (drought)}
			\end{itemize}
  
		\end{itemize}
  
	\end{itemize}
  
  \end{frame}
  
  \begin{frame}{Empirical Approach}
	\begin{itemize}
	  \item \textbf{Descriptive:} Using household surveys, measure shock incidence, estimate household responses to shocks.
	  \bigskip 
	  
	  \item \textbf{Causal impact of weather shocks:} Match remote sensing and meteorological data with household/neighborhood/municipality panel data. 	
	  \bigskip 

	  \item \textbf{Causal impact of idiosyncratic shocks:}  administrative data (SISBEN, PILA, RIPS) to identify exogenous variation in shock exposure.
	\end{itemize}
  \end{frame}

\end{document}
%   \begin{frame}{Data}
% 	\begin{itemize}
% 	  \item \textbf{Household Surveys:}
% 	  \begin{itemize}
% 		\item ELCA (Colombia)
% 		\item MxFLS (Mexico)
% 		\item ENAHO (Peru)
% 		\item ENAHO (Bolivia)
% 		\item PNAD-C (Brazil)
% 		\item EHPM (El Salvador)
% 		\item EPHPM (Honduras)
% 	  \end{itemize}
% 	  \bigskip

% 	  \item \textbf{Administrative Data:} SISBEN, PILA, RIPS
% 	  \bigskip 

% 	  \item \textbf{Remote sensing \& meteorological data:} 
% 	  \begin{itemize}
% 			\item e.g. \emph{SALURBAL} dataset
% 	  \end{itemize}
% 	\end{itemize}
%   \end{frame}
  
% \begin{frame}{SALURBAL: Neighborhood-level drought events}
%   \begin{minipage}{0.48\textwidth}
% 	\centering
% 	\includegraphics[scale=0.17]{out/nature_paper.png}
% \end{minipage}
% \hfill
% \begin{minipage}{0.48\textwidth}
% 	\centering
% 	\includegraphics[scale=0.2]{out/map_floods.png}
% \end{minipage}
% \end{frame}

%   \begin{frame}{Preliminary Hypotheses}
% 	\begin{itemize}
% 	  \item Hazard rate of shocks is higher for poorer households.
% 	  \medskip
% 	  \item Poorer households have fewer coping mechanisms.
% 	  \bigskip
% 	  \begin{itemize}
% 	  	\item Can we substantiate these claims empirically?
% 	  	\medskip
% 	  	\item Can we estimate the impact of government programs as mitigation tools?
% 	  \end{itemize}
% 	\end{itemize}
%   \end{frame}

% \begin{frame}[c]{}         
% 	\centering
% 	\large Incidence of self-reported shocks: 
% \end{frame}

% \begin{frame}{Self-reported shocks: ELCA}

% 	\begin{figure}
% 		\includegraphics[scale=0.3]{out/elca_shocks_incidence.png}
%    \end{figure}   

% \end{frame}

% \begin{frame}{Self-reported shocks: rural vs urban}

% 	\begin{figure}
% 		\includegraphics[scale=0.3]{out/elca_shocks_incidence_ruralurban.png}
%    \end{figure}   

% \end{frame}

% \begin{frame}{Self-reported shocks: rural vs urban}

% 	\begin{figure}
% 		\includegraphics[scale=0.3]{out/elca_anyshock_incidence.png}
%    \end{figure}   

% \end{frame}

% \begin{frame}{Self-reported shocks by consumption quintiles}

% 	\begin{figure}
% 		\includegraphics[scale=0.3]{out/elca_anyshock_baselinecons_ruralurban.png}
%    \end{figure}   

% \end{frame}

% \begin{frame}{Self-reported shocks by consumption quintiles - Rural}

% 	\begin{figure}
% 		\includegraphics[scale=0.3]{out/elca_shocks_baselinecons_rural.png}
%    \end{figure}   

% \end{frame}

% \begin{frame}{Self-reported shocks by consumption quintiles - Urban}

% 	\begin{figure}
% 		\includegraphics[scale=0.3]{out/elca_shocks_baselinecons_urban.png}
%    \end{figure}   

% \end{frame}

% \begin{frame}{Self-reported shocks by consumption quintiles - Average all waves}

% 	\begin{figure}
% 		\includegraphics[scale=0.3]{out/elca_anyshock_baselinecons_allysurban.png}
%    \end{figure}   

% \end{frame}

% \begin{frame}[c]{}        
% 	\centering
% 	\large Response to shocks:
% \end{frame}
  
% \begin{frame}{Different responses by shock type}

% 	\begin{align*} 
% 	    \Delta y_{i, t} = \beta Shock_{i,t} + \delta_{t} + \alpha \mathbbm{1}(rural_{i,t-1}=1) + \varepsilon_{i,t}
% 	\end{align*}

%  	\centering
%  	\scalebox{0.8}{\input{out/shock_response_elca_13_16.tex}}
%  	\medskip

% \end{frame}

% \begin{frame}{Arteaga et al., (2025)}

% 	\vspace{-20pt}
% 	\begin{align*} 
% 	y_{i, m, t} = \beta TempShock_{i,t} + X'_{m,t}\gamma + \delta_{t} + \alpha_{i} + \varepsilon_{i,t}
% 	\end{align*}

%    \vspace{-10pt}
%    \begin{figure}
% 		\includegraphics[scale=0.25]{out/farmsize_tab4.png}
%    \end{figure}   

% \end{frame}

% \begin{frame}{Different responses by household type}

% 	\vspace{-20pt}
% 	\begin{align*} 
% 	    \Delta y^{q}_{i, t} = \beta Shock_{i,t} + \delta_{t} + \alpha \mathbbm{1}(rural_{i,t-1}=1) + \varepsilon_{i,t}
% 	\end{align*}
	
% 	\vspace{-10pt}
%    \begin{figure}
% 		\includegraphics[scale=0.3]{out/elca_migrate_natdisast_baselinecons.png}
%    \end{figure}   

% \end{frame}

% \begin{frame}{Cross-country analysis:}

% 	\begin{figure}
% 		\caption{Self Reported Shoks: ENAHO - PERU}
% 		\includegraphics[scale=0.45]{out/enaho_shocks_incidence_ruralurban.png}
%    \end{figure}   

% \end{frame}


% \begin{frame}{Cross-country analysis:}

%     \begin{minipage}{0.48\textwidth}
%         \centering
% 		ELCA (COL)
%         \includegraphics[scale=0.25]{out/elca_shocks_incidence.png}
%     \end{minipage}
%     \hfill
%     \begin{minipage}{0.48\textwidth}
%         \centering
% 		ENAHO (PER)
%         \includegraphics[scale=0.3]{out/enaho_shocks_incidence.png}
%     \end{minipage}

% \end{frame}


% \end{document}


